%% Generated by Sphinx.
\def\sphinxdocclass{report}
\documentclass[letterpaper,10pt,english,openany,oneside]{sphinxmanual}
\ifdefined\pdfpxdimen
   \let\sphinxpxdimen\pdfpxdimen\else\newdimen\sphinxpxdimen
\fi \sphinxpxdimen=.75bp\relax

\PassOptionsToPackage{warn}{textcomp}
\usepackage[utf8]{inputenc}
\ifdefined\DeclareUnicodeCharacter
% support both utf8 and utf8x syntaxes
  \ifdefined\DeclareUnicodeCharacterAsOptional
    \def\sphinxDUC#1{\DeclareUnicodeCharacter{"#1}}
  \else
    \let\sphinxDUC\DeclareUnicodeCharacter
  \fi
  \sphinxDUC{00A0}{\nobreakspace}
  \sphinxDUC{2500}{\sphinxunichar{2500}}
  \sphinxDUC{2502}{\sphinxunichar{2502}}
  \sphinxDUC{2514}{\sphinxunichar{2514}}
  \sphinxDUC{251C}{\sphinxunichar{251C}}
  \sphinxDUC{2572}{\textbackslash}
\fi
\usepackage{cmap}
\usepackage[T1]{fontenc}
\usepackage{amsmath,amssymb,amstext}
\usepackage{babel}



\usepackage{times}
\expandafter\ifx\csname T@LGR\endcsname\relax
\else
% LGR was declared as font encoding
  \substitutefont{LGR}{\rmdefault}{cmr}
  \substitutefont{LGR}{\sfdefault}{cmss}
  \substitutefont{LGR}{\ttdefault}{cmtt}
\fi
\expandafter\ifx\csname T@X2\endcsname\relax
  \expandafter\ifx\csname T@T2A\endcsname\relax
  \else
  % T2A was declared as font encoding
    \substitutefont{T2A}{\rmdefault}{cmr}
    \substitutefont{T2A}{\sfdefault}{cmss}
    \substitutefont{T2A}{\ttdefault}{cmtt}
  \fi
\else
% X2 was declared as font encoding
  \substitutefont{X2}{\rmdefault}{cmr}
  \substitutefont{X2}{\sfdefault}{cmss}
  \substitutefont{X2}{\ttdefault}{cmtt}
\fi


\usepackage[Bjarne]{fncychap}
\usepackage{sphinx}

\fvset{fontsize=\small}
\usepackage{geometry}


% Include hyperref last.
\usepackage{hyperref}
% Fix anchor placement for figures with captions.
\usepackage{hypcap}% it must be loaded after hyperref.
% Set up styles of URL: it should be placed after hyperref.
\urlstyle{same}
\addto\captionsenglish{\renewcommand{\contentsname}{Contents:}}

\usepackage{sphinxmessages}
\setcounter{tocdepth}{1}



\title{Nova \sphinxhyphen{} Shoot!}
\date{Nov 26, 2020}
\release{}
\author{Vincent Meunier}
\newcommand{\sphinxlogo}{\vbox{}}
\renewcommand{\releasename}{}
\makeindex
\begin{document}

\pagestyle{empty}
\sphinxmaketitle
\pagestyle{plain}
\sphinxtableofcontents
\pagestyle{normal}
\phantomsection\label{\detokenize{index::doc}}



\chapter{Welcome to \sphinxstyleemphasis{Shoot}}
\label{\detokenize{introduction:welcome-to-shoot}}\label{\detokenize{introduction:introduction}}\label{\detokenize{introduction::doc}}
This Nova award is designed to help you explore how projectiles and space affect your life each day.

\begin{sphinxadmonition}{warning}{Warning:}
When completing this Award both the youth and involved adult leaders must obey all rules of \sphinxhref{https://www.scouting.org/health-and-safety/gss/}{Safe Scouting}. This includes (1) Completing Cyber Chip prior to starting this activity and (2) \sphinxstylestrong{ALWAYS} involve at least 2 adults in all your communications with a leader, including online. If you send email to your counselor, always add the address of another adult leader or a parent/guardian. Never reply to a message sent by an adult leader unless another adult has been copied on the email. Report any issue to your parents/guardians!
\end{sphinxadmonition}


\section{Instructions}
\label{\detokenize{introduction:instructions}}\begin{enumerate}
\sphinxsetlistlabels{\arabic}{enumi}{enumii}{}{.}%
\item {} 
Identify a \sphinxstylestrong{Nova Counselor} either within your unit, district, or council.

\item {} 
This site provides you a platform for learning and you can easily follow all requirements using the navigation menu on the left.

\item {} 
Once you have identified a Counselor, you can start working on requirements.

\item {} 
The most important aspect in any scientific endeavor is to \sphinxstylestrong{properly document progress}. This will be done, here, using a google sheet as described in more details below.

\end{enumerate}


\section{Documenting your progress}
\label{\detokenize{introduction:documenting-your-progress}}\begin{enumerate}
\sphinxsetlistlabels{\arabic}{enumi}{enumii}{}{.}%
\item {} 
A template worksheet can be found \sphinxhref{https://docs.google.com/document/d/1tOlJcGxA8rKp7cc1t8yDhrckO1bbwJQjPA0HgETOAyI/edit?usp=sharing}{here}. This is a \sphinxstyleemphasis{Google document}. \sphinxstylestrong{You will not be able to modify it until you make your own copy as I will now describe for you.}

\item {} 
Once you have opened the file on google doc, go to \sphinxcode{\sphinxupquote{File}} \(\rightarrow\) \sphinxcode{\sphinxupquote{Make a Copy}}.

\item {} 
Save the file with the following name: \sphinxstyleemphasis{Nova\_shoot\_FIRSTNAME\_LASTNAME}

\item {} 
You will use that file to enter your progress and share with your counselor.

\item {} \begin{description}
\item[{You can share your own copy of the worksheet with your counselor using the following procedure.}] \leavevmode\begin{enumerate}
\sphinxsetlistlabels{\alph}{enumii}{enumiii}{}{)}%
\item {} 
Click on the SHARE button on the top\sphinxhyphen{}right.

\item {} 
Click on “get link”.

\item {} 
Send the link to your advisor.

\end{enumerate}

\end{description}

\end{enumerate}

\begin{sphinxadmonition}{note}{Note:}
This document provides you a guide to complete the Nova award! All requirements are marked with the following symbol: \(\boxed{\mathbb{REQ}\Large \rightsquigarrow}\). In addition, a number of fun \sphinxstyleemphasis{Additional Challenges} are provided in boxes for your entertainment.
\end{sphinxadmonition}


\section{If you have any question}
\label{\detokenize{introduction:if-you-have-any-question}}
Contact your counselor or your scoutmaster! If you have questions about the program, contact Dr. Meunier  by \sphinxhref{mailto:vinmeunier@gmail.com}{email}.


\section{Other Nova modules in this series}
\label{\detokenize{introduction:other-nova-modules-in-this-series}}
\begin{sphinxadmonition}{note}{More will be added, check regularly!}

\sphinxhref{https://novadtc.readthedocs.io}{\sphinxincludegraphics[scale=0.8]{{logo-dtc2_black}.png}}

\sphinxhref{https://novashoot.readthedocs.io}{\sphinxincludegraphics[scale=0.8]{{logo-shoot_black}.png}}
\end{sphinxadmonition}


\chapter{Requirement \#1: Research and Reading}
\label{\detokenize{requirement1:requirement-1-research-and-reading}}\label{\detokenize{requirement1::doc}}\begin{description}
\item[{\(\boxed{\mathbb{REQ}\Large \rightsquigarrow}\) Choose A or B or C and complete all the requirements.}] \leavevmode\begin{enumerate}
\sphinxsetlistlabels{\Alph}{enumi}{enumii}{}{.}%
\item {} 
Watch about three hours total of science\sphinxhyphen{}related shows or documentaries that involve projectiles, aviation, weather, astronomy, or space technology. Then do the following:
\begin{enumerate}
\sphinxsetlistlabels{\arabic}{enumii}{enumiii}{(}{)}%
\item {} 
Make a list of at least five questions or ideas from the show(s) you watched.

\item {} 
Discuss two of the questions or ideas with your counselor.

\end{enumerate}

\begin{sphinxadmonition}{tip}{Tip:}
Some examples include—but are not limited to—shows found on PBS (“NOVA”), Discovery Channel, Science Channel, National Geographic Channel, TED Talks (online videos), and the History Channel. You may choose to watch a live performance or movie at a planetarium or science museum instead of watching a media production. You may watch online productions with your counselor’s approval and under your parent’s or guardian’s supervision.
\end{sphinxadmonition}

\item {} 
Research (about three hours total) several websites (with your parent’s or guardian’s permission) that discuss and explain cryptography or the discoveries of people who worked extensively with cryptography. Then do the following:Read (about three hours total) about projectiles, aviation, space, weather, astronomy, or space technology. Then do the following:
\begin{enumerate}
\sphinxsetlistlabels{\arabic}{enumii}{enumiii}{(}{)}%
\item {} 
Make a list of at least two questions or ideas from each article

\item {} 
Discuss two of the questions or ideas with your counselor

\end{enumerate}

\begin{sphinxadmonition}{tip}{Tip:}
Examples of magazines include—but are not limited to—Odyssey, Popular Mechanics, Popular Science, Science Illustrated, Discover, Air \& Space, Popular Astronomy, Astronomy, Science News, Sky \& Telescope, Natural History, Robot, Servo, Nuts and Volts, and Scientific American.
\end{sphinxadmonition}

\item {} 
Do a combination of reading and watching (about three hours total). Then do the following:
\begin{enumerate}
\sphinxsetlistlabels{\arabic}{enumii}{enumiii}{(}{)}%
\item {} 
Make a list of at least two questions or ideas from each article or show

\item {} 
Discuss two of the questions or ideas with your counselor.

\end{enumerate}

\begin{sphinxadmonition}{tip}{Tip:}
Examples of magazines include—but are not limited to—Odyssey, Popular Mechanics, Popular Science, Science Illustrated, Discover, Air \& Space, Popular Astronomy, Astronomy, Science News, Sky \& Telescope, Natural History, Robot, Servo, Nuts and Volts, and Scientific American.
\end{sphinxadmonition}

\end{enumerate}

\end{description}

\begin{sphinxadmonition}{note}{Additional Challenge}

The University of Montana has an fun crossword puzzle related to Space Exploration. Do you think you can solve it?

You can find it here as a free \sphinxhref{http://solar.physics.montana.edu/spot/teacher\_area/AA/K-4/space\%20crossword\%20puzzle.pdf}{PDF file}.

Do not look beyond page 2 until you are ready to see the solution!
\end{sphinxadmonition}

\begin{sphinxadmonition}{attention}{Attention:}
Once you have completed this requirement, make sure you document it in your worksheet!
\end{sphinxadmonition}


\chapter{Requirement \#2: Merit Badge}
\label{\detokenize{requirement2:requirement-2-merit-badge}}\label{\detokenize{requirement2::doc}}
\(\boxed{\mathbb{REQ}\Large \rightsquigarrow}\) Complete ONE merit badge from the following list. Choose one that you have not already used toward another Nova award.
After completion, discuss with your counselor how the merit badge you earned uses science and projectiles.
\begin{itemize}
\item {} 
Archery

\item {} 
Astronomy

\item {} 
Athletics

\item {} 
Aviation

\item {} 
Game Design

\item {} 
Rifle Shooting

\item {} 
Robotics

\item {} 
Shotgun Shooting

\item {} 
Space Exploration

\item {} 
Sustainability

\item {} 
Weather

\end{itemize}

\begin{figure}[htbp]
\centering

\noindent\sphinxincludegraphics[width=700\sphinxpxdimen]{{meritbadges}.png}
\end{figure}

\begin{sphinxadmonition}{attention}{Attention:}
Once you have completed this requirement, make sure you document it in your worksheet!
\end{sphinxadmonition}


\chapter{Requirement \#3: Projectile motion}
\label{\detokenize{requirement3:requirement-3-projectile-motion}}\label{\detokenize{requirement3::doc}}
\(\boxed{\mathbb{REQ}\Large \rightsquigarrow}\) Choose A or B and complete ALL the requirements.
\begin{enumerate}
\sphinxsetlistlabels{\Alph}{enumi}{enumii}{}{.}%
\item {} 
Simulations. Find and use a projectile simulation applet on the Internet (with your parent’s or guardian’s permission). Then design and complete a hands\sphinxhyphen{}on experiment to demonstrate projectile motion.
\begin{enumerate}
\sphinxsetlistlabels{\arabic}{enumii}{enumiii}{(}{)}%
\item {} 
Keep a record of the angle, time, and distance.

\item {} 
How does your horse power compare to the horse power of your favorite car? Graph the results of your experiment. (Note: Using a high\sphinxhyphen{}speed camera or
video camera may make the graphing easier, as will doing many repetitions using variable heights from which the projectile can be launched.)

\item {} 
Discuss with your counselor:
\begin{enumerate}
\sphinxsetlistlabels{\alph}{enumiii}{enumiv}{(}{)}%
\item {} 
What a projectile is.

\item {} 
What projectile motion is.

\item {} 
The factors affecting the path of a projectile.

\item {} 
The difference between forward velocity and acceleration due to gravity.

\end{enumerate}

\end{enumerate}

\end{enumerate}

\begin{sphinxadmonition}{tip}{Tip:}
Helpful Links:

Be sure you have your parent’s or guardian’s permission before using the Internet. Some of these websites require the use of Java runtime environments. If your computer does not support this program, you may not be able to visit those sites.
\begin{itemize}
\item {} 
Projectile Motion Applets: \sphinxhref{http://galileoandeinstein.physics.virginia.edu/more\_stuff/Applets/Projectile/projectile.html}{access here}.

\item {} 
Fowler’s Physics Applets Website: \sphinxhref{https://www.compadre.org/introphys/items/detail.cfm?ID=7823\#:-:text=Fowler\%27s\%20Physics\%20Applets\%20This\%20is\%20a\%20collection\%20of,collection\%20includes\%20materiaIs\%20in\%20mechanics\%2C\%20and\%20modern\%20physics}{link}.

\item {} 
Java Applets on Physics: Click \sphinxhref{http://www.cco.caltech.edu/~phys1/java.html}{here}.

\end{itemize}
\end{sphinxadmonition}
\begin{enumerate}
\sphinxsetlistlabels{\Alph}{enumi}{enumii}{}{.}%
\setcounter{enumi}{1}
\item {} 
Discover. Explain to your counselor the difference between escape velocity (not the game), orbital velocity, and terminal velocity. Then answer TWO of the following questions. (With your parent’s or guardian’s permission, you may wish to explore websites to find this information.)
\begin{enumerate}
\sphinxsetlistlabels{\arabic}{enumii}{enumiii}{(}{)}%
\item {} 
Why are satellites usually launched toward the east, and what is a launch window?

\item {} 
What is the average terminal velocity of a skydiver? (What is the fastest you would go if you were to jump out of an airplane?)

\item {} 
How fast does a bullet, baseball, airplane, or rocket have to travel in order to escape Earth’s gravitational field? (What is Earth’s escape velocity?)

\end{enumerate}

\end{enumerate}

\begin{figure}[htbp]
\centering
\capstart

\noindent\sphinxincludegraphics[width=600\sphinxpxdimen]{{f03f1c518f43e23b6c1075958acfa234f907c2ab}.jpg}
\caption{Image obtained from  National Geographic website.}\label{\detokenize{requirement3:id1}}\end{figure}

\begin{sphinxadmonition}{attention}{Attention:}
Once you have completed this requirement, make sure you document it in your worksheet!
\end{sphinxadmonition}


\chapter{Requirement \#4: Observation}
\label{\detokenize{requirement4:requirement-4-observation}}\label{\detokenize{requirement4::doc}}
\(\boxed{\mathbb{REQ}\Large \rightsquigarrow}\) Choose A or B and complete ALL the requirements.
\begin{enumerate}
\sphinxsetlistlabels{\Alph}{enumi}{enumii}{}{.}%
\item {} 
Visit an observatory or a flight, aviation, or space museum.
\begin{enumerate}
\sphinxsetlistlabels{\arabic}{enumii}{enumiii}{(}{)}%
\item {} 
During your visit, talk to a docent or person in charge about a science topic related to the site.

\item {} 
Discuss your visit with your counselor.

\end{enumerate}

\item {} 
Discover the latitude and longitude coordinates of your current position. Then do the following:
\begin{enumerate}
\sphinxsetlistlabels{\arabic}{enumii}{enumiii}{(}{)}%
\item {} 
Find out what time a satellite will pass over your area. (A good resource to find the times for satellite passes is the Heavens Above website at www.heavens\sphinxhyphen{}above.com.)

\item {} 
Watch the satellite using binoculars. Record the time of your viewing, the weather conditions, how long the satellite was visible, and the path of the satellite. Then discuss your viewing with your counselor.

\end{enumerate}

\end{enumerate}

\begin{sphinxadmonition}{tip}{Tip:}
You can complete this requirement using a \sphinxstyleemphasis{virtual} visit!
\end{sphinxadmonition}

\begin{sphinxadmonition}{note}{Additional Challenge}

Go to this \sphinxhref{https://www.funtrivia.com/playquiz/quiz20845717de740.html}{page} where a trivia is proposed with 10 somewhat challenging questions related to satellites. How many questions can you get right? If you are curious check out some of the other \sphinxstyleemphasis{fun trivias} this site offers for free!
\end{sphinxadmonition}

\begin{figure}[htbp]
\centering
\capstart

\noindent\sphinxincludegraphics[width=600\sphinxpxdimen]{{kitts-peak-observatory}.jpg}
\caption{Kitts Peak National Observatory. Image from cntravel.com, find more information on observatories on their \sphinxhref{https://www.cntraveler.com/galleries/2014-03-18/amazing-observatories-you-can-visit}{website}.}\label{\detokenize{requirement4:id1}}\end{figure}

\begin{sphinxadmonition}{attention}{Attention:}
Once you have completed this requirement, make sure you document it in your worksheet!
\end{sphinxadmonition}


\chapter{Requirement \#5: Building a machine}
\label{\detokenize{requirement5:requirement-5-building-a-machine}}\label{\detokenize{requirement5::doc}}
\(\boxed{\mathbb{REQ}\Large \rightsquigarrow}\) Choose A or B or C and complete ALL the requirements.
\begin{enumerate}
\sphinxsetlistlabels{\Alph}{enumi}{enumii}{}{.}%
\item {} 
Design and build a catapult that will launch a marshmallow a distance of 4 feet. Then do the following:
\begin{enumerate}
\sphinxsetlistlabels{\arabic}{enumii}{enumiii}{(}{)}%
\item {} 
Keep track of your experimental data for every attempt. Include the angle of launch and the distance projected.

\item {} 
Make sure you apply the same force each time, perhaps by using a weight to launch the marshmallow. Discuss your design, data, and experiments—both successes and failures—with your counselor.

\end{enumerate}

\item {} 
Design a pitching machine that will lob a softball into the strike zone. Answer the following questions, and discuss your design, data, and experiments—both successes and failures—with your counselor.
\begin{enumerate}
\sphinxsetlistlabels{\arabic}{enumii}{enumiii}{(}{)}%
\item {} 
At what angle and velocity will your machine need to eject the softball in order for the ball to travel through the strike zone from the pitcher’s mound?

\item {} 
How much force will you need to apply in order to power the ball to the plate?

\item {} 
If you were to use a power supply for your machine, what power source would you choose and why?

\end{enumerate}

\item {} 
Design and build a marble run or roller coaster that includes an empty space where the marble has to jump from one part of the chute to the other. Do the following, then discuss your design, data, and experiments—both successes and failures—with your counselor.
\begin{enumerate}
\sphinxsetlistlabels{\arabic}{enumii}{enumiii}{(}{)}%
\item {} 
Keep track of your experimental data for every attempt. Include the vertical angle between the two parts of the chute and the horizontal distance between the two parts of the chute.

\item {} 
Experiment with different starting heights for th emarble. How do the starting heights affect the velocity of the marble? How does a higher starting height affect the jump distance?

\end{enumerate}

\end{enumerate}

\begin{sphinxadmonition}{attention}{Attention:}
Once you have completed this requirement, make sure you document it in your worksheet!
\end{sphinxadmonition}

\begin{figure}[htbp]
\centering
\capstart

\noindent\sphinxincludegraphics[width=600\sphinxpxdimen]{{3d73d6ac7f06107ca4c07fad1806ab054c9bc79e}.jpg}
\caption{Image of a catapult obtained from \sphinxhref{https://www.britannica.com/technology/catapult-military-weaponry}{Encyclopedia Britannica}.}\label{\detokenize{requirement5:id1}}\end{figure}


\chapter{Requirement \#6: Science@Life}
\label{\detokenize{requirement6:requirement-6-science-life}}\label{\detokenize{requirement6::doc}}
\(\boxed{\mathbb{REQ}\Large \rightsquigarrow}\) Discuss with your counselor how math science affects your everyday life.

\begin{sphinxadmonition}{note}{Additional Challenge:}

Can you solve this challenge?
Two dogs are separated by a 2\sphinxhyphen{}mile long straight and flat road. Both start moving at 15 mph toward each other. Before they leave, a fly, which was originally on the head of the first dog, flies away towards the other dog at a constant velocity of 25 mph (this is a very fast fly!). Once it arrives at the second dog, the fly turns around and goes back towards the first dog, still at 25 mph. The fly keeps doing this until the two dogs meet.

\sphinxstylestrong{Question:} What is the total distance the fly flew during this process?
\end{sphinxadmonition}

\begin{sphinxadmonition}{attention}{Attention:}
Once you have completed this requirement, make sure you document it in your worksheet!
\end{sphinxadmonition}


\chapter{About the author}
\label{\detokenize{contact:about-the-author}}\label{\detokenize{contact::doc}}
These pages were written by Vincent Meunier, the Chair of the STEM committee of \sphinxhref{https://www.trcscouting.org}{Twin Rivers Council} in New York State.

Vincent Meunier is a Professor of physics at Rensselaer Polytechnic Institute. If you have any question, feel free to contact him by \sphinxhref{mailto:vinmeunier@gmail.com}{email}.

\begin{sphinxadmonition}{note}{Note:}
Most of the material used here was obtained from a number of external scouting sources, including \sphinxhref{https://www.scouting.org/wp-content/uploads/2018/11/Designed-to-Crunch-Nova-2018Nov26.pdf}{scouting.org}
\end{sphinxadmonition}

\begin{figure}[htbp]
\centering
\capstart

\noindent\sphinxincludegraphics[width=600\sphinxpxdimen]{{rocket-1}.jpg}
\caption{It is not rocket science \sphinxhyphen{} it is harder! Image obtained from storm\sphinxhyphen{}asia.com}\label{\detokenize{index:id1}}\end{figure}



\renewcommand{\indexname}{Index}
\printindex
\end{document}